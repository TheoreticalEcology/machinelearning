\documentclass[a4paper,twoside]{tufte-book}\usepackage[]{graphicx}\usepackage[]{color}
% maxwidth is the original width if it is less than linewidth
% otherwise use linewidth (to make sure the graphics do not exceed the margin)
\makeatletter
\def\maxwidth{ %
  \ifdim\Gin@nat@width>\linewidth
    \linewidth
  \else
    \Gin@nat@width
  \fi
}
\makeatother

\usepackage{Sweavel}

 %style file is in the same folder.

\usepackage[T1]{fontenc}
\usepackage[utf8]{inputenc}
%\usepackage{german}

\usepackage{color}
\usepackage{xcolor}
\usepackage{framed}
\usepackage{listings}

\usepackage{graphicx}

\usepackage{multicol}              
\usepackage{multirow}
\usepackage{booktabs}
\usepackage{natbib}

\usepackage[innerrightmargin = 0.7cm, innerleftmargin = 0.3cm]{mdframed}
\usepackage{mdwlist}

\usepackage[]{hyperref}
\definecolor{darkblue}{rgb}{0,0,.5}
\hypersetup{colorlinks=true, breaklinks=true, linkcolor=darkblue, menucolor=darkblue, urlcolor=blue, citecolor=darkblue}

\usepackage[toc,page]{appendix}




\setcounter{secnumdepth}{1}
\setcounter{tocdepth}{1}

\lstset{ % settings for listings needs to be be changed to R sytanx 
language=R,
breaklines = true,
columns=fullflexible,
breakautoindent = false,
%basicstyle=\listingsfont, 
basicstyle=\ttfamily \scriptsize,
keywordstyle=\color{black},                          
identifierstyle=\color{black},
commentstyle=\color{gray},
xleftmargin=3.4pt,
xrightmargin=3.4pt,
numbers=none,
literate={*}{{\char42}}1
         {-}{{\char45}}1
         {\ }{{\copyablespace}}1
}
% http://www.monperrus.net/martin/copy-pastable-listings-in-pdf-from-latex
\usepackage[space=true]{accsupp}
% requires the latest version of package accsupp
\newcommand{\copyablespace}{
    \BeginAccSupp{method=hex,unicode,ActualText=00A0}
\ %
    \EndAccSupp{}
}









\title{Machine Learning}
\author{Maximilian Pichler, Florian Hartig}


\begin{document}
%\SweaveOpts{concordance=TRUE} % don't activate this for knitr

\let\cleardoublepage\clearpage % No empty pages between chapters
\maketitle


\thispagestyle{empty}
\null


\href{Prof. Dr. Florian Hartig}{http://www.uni-regensburg.de/biologie-vorklinische-medizin/theoretische-oekologie/mitarbeiter/hartig/index.html}\\
University of Regensburg\\
Germany\\[0.5cm]

\begin{fullwidth}
Vorlesungsunterlagen für Studierende der

\begin{itemize*}
  \item BSc Biostatistik
  \item MSc Einführung in die Statistik mit R
\end{itemize*}

\vspace{0.5cm}

Fehler oder Verbesserungsvorschläge bitte über den \href{https://github.com/TheoreticalEcology/machinelearning/issues}{issue tracker} melden. 

\end{fullwidth}


\vfill
\begin{fullwidth}
Grundlagen der Statistik Version 0.1.2, compiled \today. Translated into German 2017, based on lecture notes "Essential Statistics", created 2014-2016. This work is licensed under a \href{https://creativecommons.org/licenses/by-nc-nd/4.0/}{Creative Commons Attribution-NonCommercial-NoDerivatives 4.0 International License}. 
\end{fullwidth}


\newpage
\tableofcontents

\chapter{Introduction} % Use chapters instead of sections

%\section{Goals of statistical inference}
\section{Explainatory modelling}


\section{Predictive modelling}


\section{Machine learning models}


\section{Introduction to Keras / Tensorflow}

\href{https://www.tensorflow.org/}{TensorFlow} is a open source linear algebra library with a focus on neural networks. 
%
Google published opfficialy TensorFlow (TF) in 2015. 
%
TF supports several interesting features:
%
\begin{itemize}
\item Automatic differentation (analytic)
\item Several gradient optimiziers
\item CPU and GPU parallelization
\end{itemize}
%
All operations in TF are written in C++ and are highly optimized. 
%
But dont worry, we don't have to use C++ to use TF because there are several bindings for other languages.
%
TensorFlow officialy supports a Python API, but meanwhile there are several community carried APIs for other languages:
%
\begin{itemize}
\item R
\item Go
\item Rust
\item Swift
\item JavaScript
\end{itemize}
%
In this course we will use TF with the \href{R}{https://tensorflow.rstudio.com/} binding, that was developed and published 2017 by the RStudio Team.
%
They developed first a R package (\href{https://rstudio.github.io/reticulate/}{reticulate}) to call python in R. 
%
Actually, we are using in R the python TF module (more about this later). \\
%
TF offers different levels of API.
%
We could implement a neural network completly by ourselves, or we could use Keras which is provided by TF as a submodule.
%
Keras is a powerful module for building and training neural networks.
%
It allows us to build and train neural networks in a few lines of codes.
%
Since the end of 2018, Keras and TF are completly interoperable, allowing us to utilize the best of both.
%
In this course, we will show how we can use Keras for neural networks but also how we can use the TF's automatic differenation for using complex objective functions. 
%

\subsection{TensorFlow installation}
For the TF-R binding there are several ways to install TensorFlow. 
%
The easiest way is to the follow the instructions on the RStudio-TF homepage \href{https://tensorflow.rstudio.com/tensorflow/articles/installation.html}{RStudio-TF homepage}.
%
\textbf{However}, in this course we will use TensorFlow 2.0 API, which is the new, reworked API, and yet not stable or supported by the TF-R installer.   
%
To install TF-2.0, follow the OS specific instructions below:
%
\begin{itemize}
\item \hyperref[Mac]{MacOS and Linux}
\item \hyperref[Windows]{Windows}
\end{itemize}

\paragraph{MacOS and Linux}\label{Mac} 
On Linux and and MacOS, python 2 or python 3 is already pre-installed.
%
First, we have to check whether pip (python2 package installer) or pip3 (python3 package installer) is installed. 
%
Open the terminal and hit:
\begin{Schunk}
\begin{Sinput}
which pip 
which pip3
\end{Sinput}
\end{Schunk}
If one is successfull, skip the next step, otherwise we have to install first pip now:
%
\textbf{MacOS: }
\begin{Schunk}
\begin{Sinput}
sudo easy_install pip
\end{Sinput}
\end{Schunk}
%

\textbf{Linux (Debian, Ubuntu here): }
\begin{Schunk}
\begin{Sinput}
sudo apt-get install python-pip
\end{Sinput}
\end{Schunk}
%
Now open R or execute the command in the shell for installing TF:
\begin{Schunk}
\begin{Sinput}
system("pip install --user tensorflow==2.0.0-beta0")
\end{Sinput}
\end{Schunk}



\paragraph{Windows}\label{Windows} 
On Windows, python is not pre-installed. 
%
First, download and install python2 or python3 from the python website: \href{https://www.python.org/downloads/windows/}{https://www.python.org/downloads/windows/}. 
%
\textbf{Warning:} The installer will ask you if it should set the paths for python, set the checkmark there!
%
Download the \href{https://bootstrap.pypa.io/get-pip.py}{get-pip.py} script and run in the powershell (as admin):
%
\begin{Schunk}
\begin{Sinput}
python get-pip.py
\end{Sinput}
\end{Schunk}
%
Afterwards install TF in the powershell with:
\begin{Schunk}
\begin{Sinput}
pip install --user tensorflow==2.0.0-beta0
\end{Sinput}
\end{Schunk}
%
or in R with:
\begin{Schunk}
\begin{Sinput}
system("pip install --user tensorflow==2.0.0-beta0")
\end{Sinput}
\end{Schunk}


After we have installed the TF-python package, we have to setup R for TF:
\begin{Schunk}
\begin{Sinput}
install.packages(c("reticulate", "keras"))
\end{Sinput}
\end{Schunk}
Try it out:
\begin{Schunk}
\begin{Sinput}
library(tensorflow)
# or: tf <- reticulate::import('tensorflow')
print(tf$version)
\end{Sinput}
\begin{Soutput}
Module(tensorflow._api.v2.version)
\end{Soutput}
\end{Schunk}



% I changed the order, we should first introduce the models and then data analysis workflow, otherwise we dont have something to work with....
\chapter{Important ML models}

\section{BRTs and RF}
%
Random Forest \citep[RF,][]{Breiman2001a} and Boosted Regression Trees \citep[BRT,][]{Friedman2001} are both based on the Classification And Regression Tree (CART) algorithm.
%
\subsection{CART}
Given $Y \thicksim X_1 + X_2$, the CART tries to partition the feature space into rectangulars.




\section{SVMs}

\section{ANNs and DNNs}

\section{CNNs}



\chapter{The data analysis workflow}

\section{Data prepartion}

\section{Model fitting}

\section{Model validation}

\section{Inference}



\chapter{Advanced topics}


\chapter{Resources}


\bibliographystyle{chicago}
\bibliography{ML_bib}
\end{document}
